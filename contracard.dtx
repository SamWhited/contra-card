% \iffalse
%<*internal>
\begingroup
\input docstrip.tex
\keepsilent
\usedir{tex/latex/contracard}
\preamble

  Copyright 2012 Samuel Whited

  This file may be distributed and/or modified under the
  conditions of the LaTeX Project Public License, either
  version 1.3c of this license or (at your option) any later
  version. The latest version of this license is in:

  http://www.latex-project.org/lppl.txt

  and version 1.3c or later is part of all distributions of
  LaTeX version 2008/05/04 or later.

\endpreamble
\postamble

  ___________
  Maintainer: Sam Whited
  Website:    https://samwhited.com
  Contact:    sam@samwhited.com
  Public key: 0xEC2C9934

  This work consists of this file contracard.dtx
            and the derived files contracard.cls
                              and contracard.pdf

\endpostamble
\askforoverwritefalse

\generate{\file{contracard.cls}{\from{contracard.dtx}{contracard}}}
\generate{\file{contracard-style.sty}{\from{contracard.dtx}{doc-style}}}

\obeyspaces
\Msg{****************************************************}
\Msg{*                                                  *}
\Msg{* To finish the installation you have to move the  *}
\Msg{* following file into a directory searched by TeX: *}
\Msg{*                                                  *}
\Msg{* contracard.cls                                   *}
\Msg{*                                                  *}
\Msg{****************************************************}

\endgroup
%</internal>
%
%<*driver>
\ProvidesFile{contracard.dtx}
%</driver>
%
%<*contracard>
\NeedsTeXFormat{LaTeX2e}[1994/06/01]
\ProvidesClass{contracard}[2013/02/07]
\AtEndOfClass{\LoadClass{article}}
%</contracard>%
%
%<*driver>
\documentclass[a4paper]{ltxdoc}
\usepackage{contracard-style}
\EnableCrossrefs
\CodelineIndex
\RecordChanges
\begin{document}
  \DocInput{\jobname.dtx}
\end{document}
%</driver>
%
% \fi
%
% \GetFileInfo{contracard.dtx}
% \makeatletter
% \errorcontextlines=999
%
% \title{The \textsf{contracard} class}
% \author{
%   \name{Sam Whited}\\
%   \texttt{sam@samwhited.com}
% }
% \date{\today\\\gittag}
%
% \maketitle
% \tableofcontents
%
% \section{Introduction}
%
% The \pkg{contracard} class is designed to aid in the formatting and
% typesetting of caller cards for traditional square and line dances.
%
% \subsection{Rationale}
%
% There is little to no purpose for this class. I was tired of seeing caller
% cards that were written in difficult to read shorthand, or, when they were
% presented in a digital format, using some ugly 90s-looking HTML or MS Word
% template. This is my attempt to make calling cards more beautiful.
%
% \subsection{License / about the source}
%
% This project may be distributed and/or modified under the conditions of the
% \LaTeX\ Project Public License, either version 1.3c of this license or (at
% your opinion) any later version. The latest version of this license is in:
%
% \url{http://www.latex-project.org/lppl.txt}
%
% and version 1.3c or later is part of all distributions of \LaTeX\ version
% 2008/05/04 or later.
%
% Contributions are welcome, and the latest development version of the project
% can always be found at \url{https://github.com/samwhited/contracard}.
%
% \StopEventually{}
% \changes{0.1}{2013/02/09}{Created boilerplate for class}
% \section{Package building and loading}
%
%    \begin{macrocode}
%<*contracard>
%    \end{macrocode}
%
% \paragraph{Building}
%
% Before building \pkg{contracard} you should verify that the following
% dependencies are installed:
%
%    \begin{macrocode}
\AtEndOfClass{\RequirePackage{intcalc}}
\AtEndOfClass{\RequirePackage{hyperref}}
\AtEndOfClass{\RequirePackage{ifthen}}
\AtEndOfClass{\RequirePackage{geometry}}
\AtEndOfClass{\usepackage{fancyhdr}}
\AtEndOfClass{\RequirePackage[compact]{titlesec}}
%    \end{macrocode}
% Once you have all the required packages, building \pkg{contracard} from source
% can be accomplished in multiple ways. If the Makefile is present running
% \code{make help} will tell you everything you need to know. To manually
% extract the files and generate the documentation simply run \code{pdflatex}
% or \XeLaTeX\ against \fname{contracard.dtx}:
%
% \begin{Verbatim}
%   pdflatex contracard.dtx
% \end{Verbatim}
%
% You can also use the \code{-{}-shell-escape} option to generate documentation
% which includes the current git commit short-hash in the version number:
%
% \begin{Verbatim}
%   pdflatex --shell-escape contracard.dtx
% \end{Verbatim}
%
% It should be noted that this option is extremely unsafe and you should only
% use it if you understand the risks.
%
% \paragraph{Options}
%
% \hypertarget{CONTRACARD:OPTIONS}{}
% \begin{macro}{small}
% \begin{macro}{medium}
% \begin{macro}{large}
% \begin{macro}{a7paper}
% The class can also be loaded with any of the following options:
%    \begin{macrocode}
\DeclareOption{small}{%
  \AtBeginDocument{%
    \titleformat{\section}{\normalsize\bfseries}{\thesection}{1em}{}%
  }
  \PassOptionsToPackage{%
    margin=0.25in,top=0.75in,paperwidth=5in,paperheight=3in%
  }{geometry}%
}
\DeclareOption{medium}{%
  \AtBeginDocument{\large}
  \PassOptionsToPackage{%
    margin=0.5in,top=0.75in,paperwidth=6in,paperheight=4in%
  }{geometry}%
}
\DeclareOption{large}{%
  \AtBeginDocument{\Large}
  \PassOptionsToPackage{%
    margin=0.75in,top=1in,paperwidth=8in,paperheight=5in%
  }{geometry}%
}
\DeclareOption{a7paper}{%
  \AtBeginDocument{%
    \titleformat{\section}{\normalsize\bfseries}{\thesection}{1em}{}%
  }
  \PassOptionsToPackage{%
    margin=5mm,top=15mm,paperwidth=105mm,paperheight=74mm%
  }{geometry}%
}
\DeclareOption*{%
  \PassOptionsToClass{\CurrentOption}{article}%
}
%    \end{macrocode}
% The \code{small} option creates a 127.0 by 76.2 mm (5 by 3 in) card.
% The \code{medium} option creates a 152.4 by 101.6 mm (6 by 4 in) card.
% The \code{large} option creates a 203.2 by 27.0 mm (8 by 5 in) card.
% The \code{a7paper} option creates a 105 by 74 mm (4.1 by 2.9 in) card.
% Any unrecognized options are passed to the \pkg{article} class.
% \end{macro}
% \end{macro}
% \end{macro}
% \end{macro}
%
% \section{Environment setup}
%
% First let's set up the basic environment. We'll start by turning off headers,
% and removing the header rule inserted by \pkg{fancyhdr} on pages that will
% have headers later:
%
%    \begin{macrocode}
\AtBeginDocument{\renewcommand{\headrulewidth}{0pt}}
\AtBeginDocument{\pagestyle{empty}}
%    \end{macrocode}
%
% \begin{macro}{dancecount}
% Next, we'll create the \code{dancecount} counter that will help us keep track
% of timing in the dance:
%    \begin{macrocode}
\newcounter{dancecount}
%    \end{macrocode}
% \end{macro}
%
% \subsection{Formatting}
%
% There are also some options to determine how the environment will be
% formatted.
%
% \begin{macro}{\defaultcontraenv}
% By default, all contra moves are wrapped in the \code{flushleft} environment.
% To change this you can renew \code{\textbackslash defaultcontraenv}.
%    \begin{macrocode}
\newcommand*{\defaultcontraenv}{flushleft}
%    \end{macrocode}
% \end{macro}
%
% \section{The Meat}
%
% Next we'll define the meat of the \pkg{contracard} package: The \code{contra}
% environment and the \code{\textbackslash move} command.
%
% \subsection{The \code{contra} environment}
%
% \begin{macro}{contra}
% The contra environment is pretty simple. It clears the page, resets the
% \code{dancecount} and displays some information about the dance.
%    \begin{macrocode}
\newenvironment{contra}[4][\defaultcontraenv]{%
  \clearpage
  \setcounter{dancecount}{0}
  \thispagestyle{fancy}
  \rhead{#4}
  \section*{#2}
  {#3}
  \newcommand*{\@contraenv}{#1}
  \begin{\@contraenv}
}{%
  \end{\@contraenv}
}
%    \end{macrocode}
% \end{macro}
%
% \subsection{The \code{\textbackslash move} command}
%
% \begin{macro}{\move}
% The next thing we need to define is the \code{\textbackslash move} command.
% This command takes in a move to display, and (optionally) the number of beats
% the move takes. If no beats are specified, it is assumed it takes up a full
% 16.
%    \begin{macrocode}
\newcommand*{\move}[2][16]{%
  \ifthenelse{\intcalcMod{\value{dancecount}}{64}=0}{A1.\ }{%
    \ifthenelse{\intcalcMod{\value{dancecount}-16}{64}=0}{A2.\ }{%
      \ifthenelse{\intcalcMod{\value{dancecount}-32}{64}=0}{B1.\ }{%
        \ifthenelse{\intcalcMod{\value{dancecount}-48}{64}=0}{B2.\ }{}%
      }%
    }%
  }%
  \addtocounter{dancecount}{#1}%
  #2%
  \ifthenelse{\intcalcMod{\value{dancecount}}{8}=0
    \and\not\intcalcMod{\value{dancecount}}{16}=0}{;}{\nopagebreak}%
  \ifthenelse{\not\intcalcMod{\value{dancecount}}{8}=0}{,}{\nopagebreak}%
  \ifthenelse{\intcalcMod{\value{dancecount}}{32}=0}{%
    \pagebreak[1]\vspace{1em}}{}%
  \ifthenelse{\intcalcMod{\value{dancecount}}{16}=0}{\newline}{}%
}
%    \end{macrocode}
% \end{macro}
%
% \section{Examples}
%
% A few examples are contained below:
%
% \subsection{Code's Compiling by Sam Whited}
%
% \begin{Verbatim}
% \begin{contra}{Code's Compiling}{Sam Whited}{Duple becket}
%   \move[8]{Neighbor dosido}
%   \move[8]{Partner see saw}
%   \move{Shadow swing}
%   \move[8]{Balance and petronella}
%   \move[8]{Balance and petronella}
%   \move[8]{Partner swing}
%   \move[8]{Left diagonal: Right and left through}
% \end{contra}
% \end{Verbatim}
%
% \subsection{Cherokee Shuffle by David Kaynor}
%
% \begin{Verbatim}
% \begin{contra}{Cherokee Shuffle}{David Kaynor}{Duple improper}
%   \move[8]{Circle left}
%   \move[8]{Neighbor dosido}
%   \move{Neighbor balance and swing}
%   \move[4]{Men allemande left \textonehalf}
%   \move[4]{Balance in a wave}
%   \move[8]{Partner swing}
%   \move[8]{Circle left \textthreequarters}
%   \move[4]{Balance the ring}
%   \move[4]{Partner california twirl}
% \end{contra}
% \end{Verbatim}
%
% Now that everything is defined we can process any
% \hyperlink{CONTRACARD:OPTIONS}{options} that were passed in by the user:
%    \begin{macrocode}
\ProcessOptions\relax
%</contracard>
%    \end{macrocode}
%
% \clearpage
% \setcounter{IndexColumns}{2}
% \phantomsection
% \addcontentsline{toc}{section}{Index}
% \PrintIndex
%
% \makeatletter
%   \renewenvironment{theglossary}{%
%   \glossary@prologue
%   \setlength\emergencystretch{5em}
%   \GlossaryParms \let\item\@idxitem \ignorespaces}{}
% \makeatother
% \clearpage
% \phantomsection
% \addcontentsline{toc}{section}{Change History}
% \PrintChanges
% \Finale
%
% \iffalse
%<*doc-style>
%    \begin{macrocode}
\ProvidesPackage{contracard-style}
\usepackage{fancyvrb,hologo,framed,multicol,url}
\usepackage[pdfborder=0, bookmarks, colorlinks=false, hidelinks]{hyperref}
\usepackage[parfill]{parskip}

\fvset{gobble=2}

\newcommand*{\gittag}{%
  \immediate\write18{%
    rm gittag.tex 2> /dev/null; (git describe --tags --dirty || echo "v0.0") 2> /dev/null > gittag.tex
  }%
  \InputIfFileExists{gittag.tex}{}{}\unskip%
  \immediate\write18{%
    rm gittag.tex 2> /dev/null
  }%
}

\newcommand\XeTeX{\hologo{XeTeX}}
\newcommand\XeLaTeX{\hologo{XeLaTeX}}
\newcommand\LuaTeX{\hologo{LuaTeX}}
\newcommand\LuaLaTeX{\hologo{LuaLaTeX}}

\newcommand*\name[1]{\textsc{#1}}
\newcommand*\fname[1]{\textsf{#1}}
\newcommand*\pkg[1]{\textsf{#1}}
\newcommand*\code[1]{\texttt{#1}}
%    \end{macrocode}
%</doc-style>
%\fi
\endinput
